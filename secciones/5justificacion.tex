\chapter{Justificación}
El ruido es un factor que ha tomado gran importancia en el país debido al aumento de quejas que se presentan ante las entidades gubernamentales. En el caso de Bogotá se estima que al mes hay 300 quejas por ruido ante la Secretaría Distrital del Medio Ambiente (SDA) \cite{Beltran2008}.

Adicionalmente, en ciudades del mundo que desde hace varios años han tratado problemáticas de ruido, la tendencia es disponer de sistemas de monitoreo ubicados en puntos estratégicos en donde se conoce, a través de mapas de ruido tradicionales, existen niveles altos de contaminación acústica \cite{Bellucci2018}. Estas estaciones entregan indicadores de ruido ambiental durante las 24 horas del día, permitiendo conocer de una mejor manera el comportamiento dinámico del ruido. 

Lo anterior, se integra al concepto de Smart Cities en donde a través del uso de WNS (Wireless Sensor Networks) y transductores de bajo costo se han venido implementado redes que capturan datos de ruido de forma continua \cite{Asensio2017}, los cuales son transmitidos a centros de almacenamiento en donde son convertidos en información. Inicialmente, esta información solo proporcionaba indicadores acústicos (Leq, L10, L90, entre otros) para describir el ruido.

Sin embargo, últimamente aplicando técnicas de aprendizaje automático es posible identificar la fuente de ruido y separarla de otras fuentes para poder realizar un análisis diferencial y establecer el aporte de cada fuente al problema de ruido \cite{Imoto}.

Con base en lo anterior, al disponer de un sistema que permita clasificar fuentes de ruido en un entorno urbano, permitiría a las entidades encargadas de tomar acciones en medida de ruido, como lo son las corporaciones autónomas regionales, centrar sus refuerzos en el diseño de medidas de mitigación para estas fuentes de ruido.