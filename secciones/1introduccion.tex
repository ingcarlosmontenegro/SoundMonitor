\chapter{Introducción}


%Ideas orden párrafo:

%1. Ruido y sus efectos en la salud

El oído humano es un órgano sensorial, el cual se mantiene activo todo el tiempo, Jimena Martinez \cite{JimenaMartinezLlorente2015}
 afirma que:
 
A diferencia de la visión, que se apaga por las noches, el oído es un sentido de alarma, que siempre está activo para detectar situaciones de peligro. Por lo tanto, el oído no se puede cerrar como se cierran los ojos cuando se duerme y siempre percibe todo lo que le llega.(P. 6).
 
Este órgano tiene la capacidad de recibir estímulos desde el exterior. Cada estimulo a los que se refiere anteriormente puede interpretarse como sonido, sobre el cual el mismo autor afirma que: 

El sonido es un cambio de presión del aire, que se mueve como una ola circular a partir de la fuente, [...] Estos cambios de presión entran en el canal auditivo, se transmiten del aire al tímpano del oído, que a su vez mueve los huesecillos del oído medio. Los huesecillos funcionan como un amplificador mecánico y pasan los movimientos al caracol, donde hacen moverse el líquido linfático que contiene Este, al moverse estimula los células ciliadas que a su vez reaccionan generando impulsos nerviosos que se envían al cerebro(P. 6).

Con base a la información anterior podemos introducir un término que es la base del contexto del problema y este trabajo, el ruido, el cual citando nuevamente al autor afirma que: ``se define como la sensación auditiva inarticulada generalmente desagradable, molesta para el oído. Técnicamente, se habla de ruido cuando su intensidad es alta, llegando incluso a perjudicar la salud humana.". De forma similar la directiva europea 2002/49/CE \cite{EUROPEO2002} en el artículo 3 define el ruido como 
``Sonido exterior no deseado o nocivo generado por las actividades humanas, incluido el ruido emitido por los medios de transporte, por el tráfico rodado, ferroviario y aéreo y por emplazamientos de actividades industriales".

Atendiendo al análisis anterior, el ruido, es un problema que afecta a la salud, en el artículo ``Hipoacusia por ruido: Un problema de salud y de conciencia pública"\cite{Ugalde2000} se afirma que:

Los principales efectos adversos sobre la salud reconocidos por la Organización Mundial de la Salud y otros organismos como la Agencia de Protección Ambiental de EEUU, y el Programa Internacional de Seguridad Química son : Efectos auditivos (discapacidad auditiva incluyendo tinnitus), dolor y fatiga auditiva, perturbación del sueño, efectos cardiovasculares, respuestas hormonales, disminución rendimiento en el trabajo y la escuela, molestia, interferencia con el comportamiento social, interferencia con la comunicación oral

A nivel urbano (la zona de estudio de este trabajo) el ruido es producido por diversas fuentes como autos, actividades de construcción, entre otros.  Un ejemplo es la ciudad de Bogotá (Colombia) que de acuerdo con la Secretaria Distrital de Ambiente\cite{Ambiente} ``En Bogotá D.C. las fuentes móviles (tráfico rodado, trafico aéreo, perifoneo) aporta el 60\% de la contaminación auditiva. El 40\% restante corresponde a las fuentes fijas (establecimientos de comercio abiertos al público, PYMES, grandes industrias, construcciones, etc)".

En este orden de ideas, se requiere tener un sistema que permita la clasificación y el análisis en tiempo real de fuentes ruidos en un ambiente urbano. 
