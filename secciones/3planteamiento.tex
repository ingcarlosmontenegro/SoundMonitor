\chapter{Planteamiento del problema}




El ruido en los últimos años ha tomado mayor importancia al ser considerado unos de los principales contaminantes del aire \cite{Murphy2014}. De hecho, estudios de la Unión Europea han determinado que el ruido es el segundo factor que más aporta a la contaminación del aire en ese continente por debajo solamente del material particulado \cite{EuropeanEnvironmentalAgency2014}. Adicionalmente, diferentes estudios estudios han demostrado los efectos en la salud a nivel fisiológicos y psicológicos que genera el ruido ambiental en las personas \cite{King2003}, \cite{Recio2016}. En estos efectos se pueden mencionar cefalea, problemas cardiovasculares, estrés, entre otros.

Es así como, el ruido es generado por fuentes sonoras las cuales aportan a la contaminación acústica, muchas de ellas debido a la actividad humana. Como fuentes de ruido en entornos urbanos se pueden mencionar dos principales categorías. La primera, corresponde a fuentes de ruido estáticas en donde industrias, bares, comercio, entre otros, generalmente, radian energía acústica de forma esférica en el ambiente. La segunda categoría corresponde a fuentes de ruido móviles que son asociadas a fuentes de transporte terrestre, aeronáutico y férreo \cite{Murphy2014}. 

En el contexto colombiano, el ruido ambiental es regulado con la resolución 627 de 2006 \cite{Ministeriodeambienteviviendaydesarrolloterritorial2006}. En ella se establece la forma como se deben realizar los estudios de ruido ambiental en el país y los niveles máximos permisibles dependiendo del uso del suelo. De igual manera, en esta resolución se determina que es obligación de las corporaciones autónomas regionales, con base en los estudios de ruido ambiental, establecer medidas para controlar sus niveles.

Adicionalmente, para realizar medidas de gestión de ruido ambiental es necesario conocer el aporte de cada fuente de ruido. Es así como, la aplicación de técnicas de aprendizaje automático para clasificar y estimar la contribución de los medios de transporte al ruido ambiental en términos de indicadores acústicos, mediante la clasificación del tráfico pesado, liviano (motos y automóviles) y aéreo es relevante. Lo anterior, toma importancia en la medida que la principal fuente de ruido en grandes ciudades suelen ser los medios de transporte \cite{omidvari2009effects}. 

Al lograr caracterizar, categorizar y separar las fuentes de ruido ambiental móviles, se permitiría conocer el aporte de cada fuente de forma independiente lo cual aportaría información a las entidades estatales encargadas de realizar los planes de gestión de ruido, para lograr niveles acústicos adecuados según el uso de suelo de un territorio. De esta manera, se plantea la siguiente pregunta problema:

¿Cómo se pueden clasificar las fuentes de ruido en un entorno urbano?
