\chapter{Glosario}
\begin{itemize}
    \item Advanced Encryption Standard (AES): ``Es un algoritmo de cifrado simétrico aplicado por NIST (Instituto Nacional de Estándares y Tecnología)"\cite{RFC3826}.
    \item Concurrencia: ``Es la tendencia de las cosas a producirse al mismo tiempo en un sistema"\cite{contconcurrency}.
    \item Fisiología: ``Ciencia que tiene por objeto el estudio de las funciones de los seres orgánicos"\cite{rae_2020}.
    \item Free Lossless Audio Codec (FLAC): ``Formato de audio similar al MP3, pero sin pérdidas, [...] el audio se comprime en FLAC sin pérdida de calidad"\cite{flac}.
    \item Frecuencia: ``Medida del número de veces que se repite un fenómeno por unidad de tiempo [...]fenómenos ondulatorios, tales como el sonido y las ondas electromagnéticas"\cite{euFrecuencia}.
    \item Edge-Computing: ``Es un tipo de informática que ocurre en la ubicación física del usuario, de la fuente de datos, o cerca de ellas"\cite{redHatEC2020}.
    \item Mel Frequency Cepstral Coefficients (MFCC): ``Representan la amplitud del espectro del habla de manera compacta, esto los ha vuelto la técnica de extracción de características más usada en reconocimiento del habla"\cite{MFCC}.
    \item Message Queue Telemetry Transport MQTT: ``Es un protocolo de mensajería estándar de OASIS para Internet de las cosas. Está diseñado como un transporte de mensajería de publicador/subscriptor extremadamente liviano"\cite{mqtt}.
    \item Programación concurrente: ``Rama de la informática que trata de las
    técnicas de programación que se usan para expresar el paralelismo entre tareas y para
    resolver los problemas de comunicación y sincronización entre procesos"\cite{concurrencia_2020}.
    \item Señal Análoga: ``Es una señal continua en la que una cantidad variable en el tiempo (como voltaje, presión, etc.) representa otra variable basada en el tiempo"\cite{arrowAnalogous}
    \item Señal digital: ``Una señal digital es una señal que representa datos como una secuencia de valores discretos"\cite{monloganalog}.
    \item Serverless: ``Es una manera de crear y ejecutar aplicaciones y servicios sin tener que administrar infraestructura [...] no tiene que aprovisionar, escalar ni mantener servidores para ejecutar aplicaciones, bases de datos y sistemas de almacenamiento" \cite{aWsSl}.
    \item Smart city: ``Ciudad que utiliza tecnología digital para conectar, proteger y mejorar la vida de los ciudadanos. Los sensores de IoT, las cámaras de video, las redes sociales y otras entradas actúan como un sistema nervioso, proporcionando al operador de la ciudad y a los ciudadanos una retroalimentación constante para que puedan tomar decisiones informadas"\cite{ciscoSC}.
    
    \item Rivest, Shamir y Adleman (RSA): ``Es un sistema de cifrado de clave pública, [...] cualquiera puede enviar un mensaje cifrado secreto a un receptor designado. Esto es sin que haya ningún contacto previo utilizando solo información disponible públicamente "\cite{mitRSA}.
    
    \item Virtual Private Network(VPN): ``Red privada que utiliza una red pública para conectar sitios o usuarios remotos entre sí. En vez de utilizar una conexión real dedicada como línea arrendada, utiliza conexiones "virtuales" enrutadas a través de Internet desde la red privada de la empresa hacia el empleado o el sitio remoto"\cite{ciscoVPN}.
    \item Visión artificial: ``Tiene como objetivo generar descripciones inteligentes y útiles de escenas y secuencias visuales, así como de los objetos que aparecen en ellas, mediante la realización de operaciones sobre imágenes y vídeos"\cite{mathworksVA}.


    
\end{itemize}